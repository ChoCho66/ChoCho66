\newcommand{\SLE}{\operatorname{SLE}}

\usepackage{booktabs}
\newenvironment{abstract}{}{}
\newenvironment{pf}{\begin{proof}}{\end{proof}}


\usepackage{fontspec}
%\usepackage{amsfonts}
\usepackage{amsthm}

%\usepackage{nicefrac} %好看的分數 專門放在指數  \nicefrac{1}{2}


\usepackage{amssymb}
% \allowdisplaybreaks[1]

%\usepackage{nath}
%\delimgrowth=1


\numberwithin{equation}{section}



\renewcommand{\baselinestretch}{1.2}  %字間的行距


\usepackage{xeCJK}
\setCJKmainfont{標楷體}
\XeTeXlinebreaklocale "zh"
\XeTeXlinebreakskip = 0pt plus 1pt
\defaultCJKfontfeatures{AutoFakeBold=6,AutoFakeSlant=.4}
%\usepackage{lipsum}
%\usepackage[most]{tcolorbox}


%\setlength{\leftmargini}{6pt}

%\newenvironment{tbox}{\begin{tcolorbox}[colback=white,enhanced jigsaw,breakable, boxrule = 0.2mm, ,left=6pt,right=6pt,top=1pt,bottom=1pt,arc = 0mm]}{\end{tcolorbox}}




\newlength{\zhind}
\settowidth{\zhind}{fives}
\AtBeginDocument{\setlength{\parindent}{\zhind}}  %每行留白距離

\AtBeginDocument{\setlength{\abovedisplayskip}{10pt}}
\AtBeginDocument{\setlength{\belowdisplayskip}{10pt}}

\AtBeginDocument{\setlength{\parskip}{.6em}}  %換段落的距離

%%%%%%%%%%%%%%%%%%%%%%%%%%%%%%%%%%%%%%%%%%%%%%%%%%%%%%%%%%%%

\newtheoremstyle{mystyle} % style name
            {3pt}         % vertical space above
            {3pt}         % vertical space below
            {}            % body font
            {}            % indent dim
            {\bfseries}   % Theorem head font
            {.}           % punctuation after theorem head
            {0.5em}       % space after theorem head
            {}            % theorem spec
%\theoremstyle{mystyle}
\newtheorem{theorem}{Theorem}[section]
%\newtheorem{theorem}{Theorem}[subsection]
\newtheorem{corollary}[theorem]{Corollary}
\newtheorem{lemma}[theorem]{Lemma}
\newtheorem{definition}[theorem]{Definition}
\newtheorem{proposition}[theorem]{Proposition}
\newtheorem{example}[theorem]{Example}
\newtheorem{remark}[theorem]{Remark}
\newtheorem{problem}[theorem]{Problem}
\newtheorem{question}[theorem]{Question}
\newtheorem{exercise}[theorem]{Exercise}
\newtheorem{conjecture}[theorem]{Remark}
\newtheorem{hypothesis}[theorem]{Hypothesis}
\newtheorem{tttttt}[theorem]{TTTTTT}


%%%%%%%%%%%%%%%%%%%%%%%%%%%%%%%%%%%%%%%%%%%%%%%%%%%%%%%%%%%%%%%
% read https://bookdown.org/yihui/rmarkdown-cookbook/latex-header.html
\usepackage{fancyhdr}   
% \fancypagestyle{plain}{%
% \fancyhf{}
% \fancyhead[L]{\leftmark}
\fancyhead[RE]{\nouppercase{\leftmark}}
\fancyhead[LO]{\nouppercase{\rightmark}}
\fancyhead[RO,LE]{\thepage}
\fancyfoot{}
% }
\renewcommand{\headrulewidth}{0pt}% Default \headrulewidth is 0.4pt




%%%%%%%%%%%%%%%%%%%%%%%%%%%%%%%%%%%%%%%%%%%%%%%%%%%%%%%%%%%%%%%


\newcommand{\lr}[1]{\left(  #1  \right)}
\newcommand{\lrr}[1]{\left[  #1  \right]}
\newcommand{\lrrr}[1]{\left\{  #1  \right\}}





%%%%%%%%%%%%%%%%%%%%%%%%
\delimitershortfall -0.1em  % 調整\left \right nested的大小
\let\originalleft\left
\let\originalright\right
\renewcommand{\left}{\mathopen{}\mathclose\bgroup\originalleft}
\renewcommand{\right}{\aftergroup\egroup\originalright}





%\usepackage{makeidx}
%\makeindex
\usepackage{imakeidx}
%\usepackage[nonewpage]{imakeidx}
%\indexsetup{level=\section*,toclevel=section,headers={Stellenregister}{\indexname}}%
%\indexsetup{level=\section*,toclevel=section,noclearpage}
\makeindex[name=term,title={Index of Terms}]
\makeindex[name=notation,title={Index of Notations}]

\usepackage[nottoc]{tocbibind}

%\usepackage[index]{kantlipsum}




% 要放在 \usepackage{imakeidx} 後面 
\makeindex


% 顯示 label !
\usepackage[left]{showlabels}
\renewcommand{\showlabelfont}{\ttfamily\tiny}


%\usepackage{mathtools} % 只列出有 eqref 的equation num  在 tinytex 會出錯
%\mathtoolsset{showonlyrefs=true}
%\mathtoolsset{showonlyrefs,showmanualtags}

%% 更改水平線長度  % footnote 會GG
%\let\oldrule=\rule
%\renewcommand{\rule}[1]{\oldrule{\linewidth}}

\usepackage{url}  % bib url 用的



%% cols

\newenvironment{columns}[1][]{}{}

\newenvironment{column}[1]{\begin{minipage}{#1}\ignorespaces}{%
\end{minipage}
\ifhmode\unskip\fi
\aftergroup\useignorespacesandallpars}

\def\useignorespacesandallpars#1\ignorespaces\fi{%
#1\fi\ignorespacesandallpars}

\makeatletter
\def\ignorespacesandallpars{%
  \@ifnextchar\par
    {\expandafter\ignorespacesandallpars\@gobble}%
    {}%
}
\makeatother



% 聖經
% \usepackage[left]{lineno}
% \linenumbers



% 圖片強制位置
\usepackage{float}

% subfig
%\usepackage{subfig}
\usepackage[margin=10pt]{subfig}
%\setlength{\subfigcapmargin}{5pt}
%\captionsetup[subfigure]{width=0.9\textwidth}

% 動畫
% \usepackage{animate}


% Do not page break for Environment
% 這樣弄的話 環境裡面不能放圖片
\AtBeginEnvironment{theorem}{\bigskip\begin{center}\begin{minipage}{\textwidth}}
\AtEndEnvironment{theorem}{\end{minipage}\end{center}\bigskip}
\AtBeginEnvironment{proposition}{\bigskip\begin{center}\begin{minipage}{\textwidth}}
\AtEndEnvironment{proposition}{\end{minipage}\end{center}\bigskip}
\AtBeginEnvironment{lemma}{\bigskip\begin{center}\begin{minipage}{\textwidth}}
\AtEndEnvironment{lemma}{\end{minipage}\end{center}\bigskip}
\AtBeginEnvironment{definition}{\bigskip\begin{center}\begin{minipage}{\textwidth}}
\AtEndEnvironment{definition}{\end{minipage}\end{center}\bigskip}
\AtBeginEnvironment{conjecture}{\bigskip\begin{center}\begin{minipage}{\textwidth}}
\AtEndEnvironment{conjecture}{\end{minipage}\end{center}\bigskip}


% pdfpages
\usepackage{pdfpages}